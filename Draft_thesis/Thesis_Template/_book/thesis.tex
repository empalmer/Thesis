% This is the Reed College LaTeX thesis template. Most of the work
% for the document class was done by Sam Noble (SN), as well as this
% template. Later comments etc. by Ben Salzberg (BTS). Additional
% restructuring and APA support by Jess Youngberg (JY).
% Your comments and suggestions are more than welcome; please email
% them to cus@reed.edu
%
% See http://web.reed.edu/cis/help/latex.html for help. There are a
% great bunch of help pages there, with notes on
% getting started, bibtex, etc. Go there and read it if you're not
% already familiar with LaTeX.
%
% Any line that starts with a percent symbol is a comment.
% They won't show up in the document, and are useful for notes
% to yourself and explaining commands.
% Commenting also removes a line from the document;
% very handy for troubleshooting problems. -BTS

% As far as I know, this follows the requirements laid out in
% the 2002-2003 Senior Handbook. Ask a librarian to check the
% document before binding. -SN

%%
%% Preamble
%%
% \documentclass{<something>} must begin each LaTeX document
\documentclass[12pt,twoside]{reedthesis}
% Packages are extensions to the basic LaTeX functions. Whatever you
% want to typeset, there is probably a package out there for it.
% Chemistry (chemtex), screenplays, you name it.
% Check out CTAN to see: http://www.ctan.org/
%%
\usepackage{graphicx,latexsym}
\usepackage{amsmath}
\usepackage{amssymb,amsthm}
\usepackage{longtable,booktabs,setspace}
\usepackage{chemarr} %% Useful for one reaction arrow, useless if you're not a chem major
\usepackage[hyphens]{url}
% Added by CII
\usepackage{hyperref}
\usepackage{lmodern}
\usepackage{float}
\floatplacement{figure}{H}
% End of CII addition
\usepackage{rotating}

% Next line commented out by CII
%%% \usepackage{natbib}
% Comment out the natbib line above and uncomment the following two lines to use the new
% biblatex-chicago style, for Chicago A. Also make some changes at the end where the
% bibliography is included.
%\usepackage{biblatex-chicago}
%\bibliography{thesis}


% Added by CII (Thanks, Hadley!)
% Use ref for internal links
\renewcommand{\hyperref}[2][???]{\autoref{#1}}
\def\chapterautorefname{Chapter}
\def\sectionautorefname{Section}
\def\subsectionautorefname{Subsection}
% End of CII addition

% Added by CII
\usepackage{caption}
\captionsetup{width=5in}
% End of CII addition

% \usepackage{times} % other fonts are available like times, bookman, charter, palatino


% To pass between YAML and LaTeX the dollar signs are added by CII
\title{Thesis}
\author{Emily Palmer}
% The month and year that you submit your FINAL draft TO THE LIBRARY (May or December)
\date{May 2018}
\division{Mathematics and Natural Sciences}
\advisor{Andrew Bray}
\institution{Reed College}
\degree{Bachelor of Arts}
%If you have two advisors for some reason, you can use the following
% Uncommented out by CII
% End of CII addition

%%% Remember to use the correct department!
\department{Mathematics}
% if you're writing a thesis in an interdisciplinary major,
% uncomment the line below and change the text as appropriate.
% check the Senior Handbook if unsure.
%\thedivisionof{The Established Interdisciplinary Committee for}
% if you want the approval page to say "Approved for the Committee",
% uncomment the next line
%\approvedforthe{Committee}

% Added by CII
%%% Copied from knitr
%% maxwidth is the original width if it's less than linewidth
%% otherwise use linewidth (to make sure the graphics do not exceed the margin)
\makeatletter
\def\maxwidth{ %
  \ifdim\Gin@nat@width>\linewidth
    \linewidth
  \else
    \Gin@nat@width
  \fi
}
\makeatother

\renewcommand{\contentsname}{Table of Contents}
% End of CII addition

\setlength{\parskip}{0pt}

% Added by CII

\providecommand{\tightlist}{%
  \setlength{\itemsep}{0pt}\setlength{\parskip}{0pt}}

\Acknowledgements{
I want to thank a few people.
}

\Dedication{
You can have a dedication here if you wish.
}

\Preface{
This is an example of a thesis setup to use the reed thesis document
class (for LaTeX) and the R bookdown package, in general.
}

\Abstract{
The preface pretty much says it all. \par

Second paragraph of abstract starts here.
}

% End of CII addition
%%
%% End Preamble
%%
%

\usepackage{amsthm}
\newtheorem{theorem}{Theorem}[chapter]
\newtheorem{lemma}{Lemma}[chapter]
\theoremstyle{definition}
\newtheorem{definition}{Definition}[chapter]
\newtheorem{corollary}{Corollary}[chapter]
\newtheorem{proposition}{Proposition}[chapter]
\theoremstyle{definition}
\newtheorem{example}{Example}[chapter]
\theoremstyle{definition}
\newtheorem{exercise}{Exercise}[chapter]
\theoremstyle{remark}
\newtheorem*{remark}{Remark}
\newtheorem*{solution}{Solution}
\begin{document}

% Everything below added by CII
  \maketitle

\frontmatter % this stuff will be roman-numbered
\pagestyle{empty} % this removes page numbers from the frontmatter
  \begin{acknowledgements}
    I want to thank a few people.
  \end{acknowledgements}
  \begin{preface}
    This is an example of a thesis setup to use the reed thesis document
    class (for LaTeX) and the R bookdown package, in general.
  \end{preface}
  \hypersetup{linkcolor=black}
  \setcounter{tocdepth}{2}
  \tableofcontents

  \listoftables

  \listoffigures
  \begin{abstract}
    The preface pretty much says it all. \par
    
    Second paragraph of abstract starts here.
  \end{abstract}
  \begin{dedication}
    You can have a dedication here if you wish.
  \end{dedication}
\mainmatter % here the regular arabic numbering starts
\pagestyle{fancyplain} % turns page numbering back on

\chapter{thesisdown::thesis\_gitbook:
default}\label{thesisdownthesis_gitbook-default}

\chapter{Chapter 1}\label{chapter-1}

\section{Introduction}\label{introduction}

Music often has subjective summary ``statistics'' throughout music
literature and everyday conversation. For example we talk about
Beethoven's Symphony 9 (think ode to joy) as ``majestic, powerful
fateful, expertly written, etc'', music critics go more in depth to
say\ldots{}. , and those with a musical background might go into more
description into the actual music compared to the emotional affect.

What if one wanted to compare Beethoven to say, Bach? One might say that
Beethoven was a classical composer, whereas Bach was a baroque composer.
What exactly to those classifiers mean? Sure, it is very well documented
the years each composer was active in, and that there were large changes
in the popular aspects of classical music. Even the untrained ear can
distinguish differences between Bach and Beethoven. What exactly is the
difference that one can hear? Does Bach follow counterpoint rules more
exactly? What ways can we empirically differentiate those two composers?
How about Mozart and Saliery, contemporaries? While those familiar with
classical music can spot the differences between the two composers, it
is more difficult for the untrained ear or eye to spot the differences.

What if we had a piece and didn't know who wrote it?

\section{Literature Review}\label{literature-review}

\subsection{Federalist Papers}\label{federalist-papers}

Mosteller and Wallace had a similar question, but regarding writing
authorship (Mosteller \& Wallace, 1964). The famous Federalist Papers
were written under the pen name `Publius'. There are several disputed
papers attributed to James Madison or Alexander Hamilton. Historians
have often examined the papers using styles of previously known writings
of Madison and Hamilton. () Using the frequency of words such as and
`by', `from', and `upon' Mosteller and Wallace trained the writings on a
set of pieces of each. These unconscious indicators were able to
differentiate between the two writers, and when a model was trained
(using \ldots{}. ), the model was able to identify the author of the
disputed paper.

\subsection{Music}\label{music}

Analyzing music, almost any piece by any composer, has already been
thoroughly examined my music historians. (Paragraph about things that
music historians talk about when analyzing music. find sources for this.
)

The human eye, no matter how well trained in music, has an extremely
hard time noticing small features throughout a piece. Even if one has a
feature in mind, would one want to count the number of times an author
used the word `as' in a 500 page book? Would one trust that count to be
accurate? Say one composer was very fond of middle C, and consistently
used it slightly more than other composers. Unless the use of C was
extreme, likely breaking rules of counterpoint and making odd melodic
choices, a human might not be able to catch this characteristic. Writing
has certain rules of grammar, that one would expect all published
writers to mostly follow. Writers would likely never have the word `as'
written twice in a row. Similarly, classical music has rules and
conventions. Counterpoint (described in Chapter 2), melody, and
characteristics of the instrument composed for constrict a composer.

How can one find similar unconscious features comparable to word
frequency for a composer? Some have analyzed these features in audio
format (MIDI) to distinguish certain genres of music. (De León \&
Inesta, 2003) They used self-organizing neural maps to classify music as
either jazz or classical. In this paper we look at music in sheet music
form. Sheet music is how composers write the music, and sheet music
contains all the the information displayed clearly that a recording
might not be obvious.

We then come to an interesting problem regarding extracting information
from sheet music. Writing such as in the federalist papers is one word
then the next, read left to right line by line, one word at a time.
However music is read in a variety of ways. It can be read left to right
note by note, but it can also be read vertically as the harmony, or the
notes played together. Also in a piece with several instruments, the
above happens at the same time for each instrument. There are also
aspects that take place over large sections, such as phrasing, or
cadencial patterns. There are rules of counterpoint that are followed
throughout the entire piece. How then do we detect what features are
those of rules and practices of classical music, and where the
creativeness and individuality of a composer happens?

Most of the musical stylometry papers have focused on composers in the
rennesanse, baroque and classical period. The Mendelssohns were
composing in the Romantic period. This choice might be because composers
in earlier eras had less ``expressive'' allowances for their composing,
thus making features easier.

\subsection{Previous choices of
features}\label{previous-choices-of-features}

Extracting features of music is the first step to analysis. Depending on
the characteristics of the composer and time period, different features
would be useful. Often, features are extracted en masse and then work is
done later to determine which features are important or useful in
identifying style.

There are two general groups of features. The first is low level, things
such as note frequency, rhythem frequency etc. The second is high level,
such as \ldots{}

Work by Backer and Kranenburg (Backer \& Kranenburg, 2005) (Should I
talk about the data they used too?) analyzed the music of Bach, Handel,
Telemann, Mozart and Haydn. They use overlapping windowing over each
entire composition to produce more data, and aviod issues of
dimensionality (?). They chose a window of 30 bars to create a high
enough number of frangments per piece and a low enough variance of the
feature values between fragments. They chose to extract 20 features.
They extracted information

A number of previous papers have focused oh Josquin des Prez. This is
likely due to the fact that there is a large training and testing data
set available in ealisly analyzable format provided by the Josquin
Research Project (citation). In addition there are a number of pieces of
disputed authorship that have been attributed to him. Work by Speiser
and Gupta (Speiser \& Gupta, n.d.) analyzed Josquin and his
contemporaries to attempt to classify unknown works. They extracted four
categories of features, frequencies of individual notes, frequencies of
pairwise interval combinations between each of the voices, markov
transition matrices for the rhythms of the pieces, and markov transition
matrices of the pitches in each piece. In total, this lead to a total of
3000 features. (Help why?)

\subsection{Methods for analysis}\label{methods-for-analysis}

Most of the previous research has needed to do some kind of feature
selection. The whole reason use machine learning is because us humans
cannot detect which features are important and distinguishing.

\section{Fanny and Felix Mendelssohn}\label{fanny-and-felix-mendelssohn}

\chapter{}\label{section}

\section{About the data and conversion
process}\label{about-the-data-and-conversion-process}

\section{About the functions}\label{about-the-functions}

\section{Methods}\label{methods}

\chapter{EDA}\label{eda}

\chapter*{Conclusion}\label{conclusion}
\addcontentsline{toc}{chapter}{Conclusion}

\chapter{The First Appendix}\label{the-first-appendix}

\chapter*{References}\label{references}
\addcontentsline{toc}{chapter}{References}

\hypertarget{refs}{}
\hypertarget{ref-backer2005}{}
Backer, E., \& Kranenburg, P. van. (2005). On musical stylometry---a
pattern recognition approach. \emph{Pattern Recognition Letters},
\emph{26}(3), 299--309.

\hypertarget{ref-de2003feature}{}
De León, P. J. P., \& Inesta, J. M. (2003). Feature-driven recognition
of music styles, 773--781.

\hypertarget{ref-mosteller1964inference}{}
Mosteller, F., \& Wallace, D. (1964). \emph{Inference and disputed
authorship: The federalist}. Addison-Wesley.

\hypertarget{ref-CompStyleAttri}{}
Speiser, J., \& Gupta, V. (n.d.). Composer style attribution.
\emph{Project Report for CS}, \emph{229}.


% Index?

\end{document}
